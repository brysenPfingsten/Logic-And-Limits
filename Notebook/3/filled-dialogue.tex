\documentclass[11pt]{article}
\usepackage[margin=1in]{geometry}
\usepackage{setspace}
\doublespacing

\begin{document}
\begin{center}
{\LARGE \textbf{Sonata for Unaccompanied Achilles}}
\end{center}

\noindent
\textbf{Achilles}: Hello, this is Achilles. 

\noindent
\textbf{Tortoise}: (Hello Achilles, it is I, the Tortoise.) 

\noindent
\textbf{Achilles}: Oh, hello, Mr. T. How are you?

\noindent
\textbf{Tortoise}: (Not very well, I’m afraid. I’ve developed a terrible case of torticollis.) 

\noindent
\textbf{Achilles}: A torticollis? Oh, I'm sorry to hear it. Do you have any idea what caused it? 

\noindent
\textbf{Tortoise}: (I believe it was from staring at the ceiling for several hours with my head cocked at a forty-five-degree angle.) 

\noindent
\textbf{Achilles}: How long did you hold it in that position? 

\noindent
\textbf{Tortoise}: (Oh, for the better part of the afternoon.) 

\noindent
\textbf{Achilles}: Well, no wonder it's stiff, then. What on earth induced you keep your neck twisted that way for so long? 

\noindent
\textbf{Tortoise}: (I was watching the clouds. I saw a most remarkable parade of creatures floating by.) 

\noindent
\textbf{Achilles}: Wondrous many of them, eh? What kinds, for example? 

\noindent
\textbf{Tortoise}: (There were giant flamingos, multi-headed hydras, and several phantasmagorical beasts.) 

\noindent
\textbf{Achilles}: What do you mean, "phantasmagorical beasts"? 

\noindent
\textbf{Tortoise}: (Creatures that don't exist in nature—beings with wings where their legs should be and eyes in their tails.) 

\noindent
\textbf{Achilles}: Wasn't it terrifying to see so many of them at the same time? 

\noindent
\textbf{Tortoise}: (Not at all; they were quite peaceful. One of them was even carrying a guitar.) 

\noindent
\textbf{Achilles}: A guitar!? Of all things to be in the midst of all those weird creatures. Say, don't you play the guitar? 

\noindent
\textbf{Tortoise}: (Actually, Achilles, I play the lute. It’s a common misconception.) 

\noindent
\textbf{Achilles}: Oh, well, it's all the same to me. 

\noindent
\textbf{Tortoise}: (Hardly! A guitar has a flat back, while a lute—or a fiddle—is vaulted.) 

\noindent
\textbf{Achilles}: You're right; I wonder why I never noticed that difference between fiddles and guitars before. Speaking of fiddling, how would you like to come over and listen to one of the sonatas for unaccompanied violin by your favorite composer, J. S. Bach? I just bought a marvelous recording of them. I still can't get over the way Bach uses a single violin to create a piece with such interest. 

\noindent
\textbf{Tortoise}: (I would love to, but this neck pain has given me a secondary affliction: a pounding headache.) 

\noindent
\textbf{Achilles}: A headache too? That's a shame. Perhaps you should just go to bed. 

\noindent
\textbf{Tortoise}: (I tried, but I can't sleep. There is a certain word puzzle stuck in my mind like a needle in a groove.)

\noindent
\textbf{Achilles}: I see. Have you tried counting sheep? 

\noindent
\textbf{Tortoise}: (Sheep are far too fluffy and indistinct. No, this puzzle is much more rigid. It’s a linguistic obsession.)

\noindent
\textbf{Achilles}: Oh, oh, I see. Yes, I fully know what you mean. Well, if it's THAT distracting, perhaps you'd better tell it to me, and let me try to work on it, too. 

\noindent
\textbf{Tortoise}: (Very well. I am looking for a common English word that contains the letters 'A', 'D', 'A', 'C' consecutively inside it.)

\noindent
\textbf{Achilles}: A word with the letters A', D', A', C' consecutively inside it ... Hmm ... What about "abracadabra"? 

\noindent
\textbf{Tortoise}: (Close, but no. That has A-D-A-B, and then later C-A-D-A.) 

\noindent
\textbf{Achilles}: True, "ADAC" occurs backwards, not forwards, in that word. 

\noindent
\textbf{Tortoise}: (Exactly. I’ve been wrestling with it for hours and hours.) 

\noindent
\textbf{Achilles}: Hours and hours? It sounds like I'm in for a long puzzle, then. Where did you hear this infernal riddle? 

\noindent
\textbf{Tortoise}: (I overheard it from a meditative-looking fellow sitting by the pond earlier today.) 

\noindent
\textbf{Achilles}: You mean he looked like he was meditating on esoteric Buddhist matters, but in reality he was just trying to think up complex word puzzles? 

\noindent
\textbf{Tortoise}: (Precisely. A passing snail told me the man was actually a retired typesetter with a grudge against the alphabet.) 

\noindent
\textbf{Achilles}: Aha!—the snail knew what this fellow was up to. But how did you come to talk to the snail? 

\noindent
\textbf{Tortoise}: (He was moving at my pace, so we had plenty of time for a chat. But back to the puzzle—it's driving me mad.) 

\noindent
\textbf{Achilles}: Say, I once heard a word puzzle a little bit like this one. Do you want to hear it? Or would it just drive you further into distraction? 

\noindent
\textbf{Tortoise}: (At this point, another riddle might act as an antidote.) 

\noindent
\textbf{Achilles}: I agree—can't do any harm. Here it is: What's a word that begins with the letters "HE" and also ends with "HE"? 

\noindent
\textbf{Tortoise}: (That’s easy. The word is "HE".) 

\noindent
\textbf{Achilles}: Very ingenious—but that's almost cheating. It's certainly not what I meant! 

\noindent
\textbf{Tortoise}: (Why not? It starts with HE and ends with HE. It's perfectly logical.) 

\noindent
\textbf{Achilles}: Of course you're right—it fulfills the conditions, but it's a sort of "degenerate" solution. There's another solution which I had in mind. 

\noindent
\textbf{Tortoise}: (Oh! You mean "HEADACHE"!) 

\noindent
\textbf{Achilles}: That's exactly it! How did you come up with it so fast? 

\noindent
\textbf{Tortoise}: (Well, as I mentioned, my head is currently throbbing, so the word was quite literally 'top of mind.') 

\noindent
\textbf{Achilles}: So here's a case where having a headache actually might have helped you, rather than hindering you. Excellent! But I'm still in the dark on your "ADAC" puzzle. 

\noindent
\textbf{Tortoise}: (Wait! I've got it! I see it now! The answer to my puzzle was right in front of me!) 

\noindent
\textbf{Achilles}: Congratulations! Now maybe you'll be able to get to sleep! So tell me, what is the solution? 

\noindent
\textbf{Tortoise}: (I'll give you a hint instead. It has to do with how you perceive the background of a drawing.) 

\noindent
\textbf{Achilles}: Well, normally I don't like hints, but all right. What's your hint? 

\noindent
\textbf{Tortoise}: (Think of the relationship between Figure and Ground. Think of M.C. Escher.) 

\noindent
\textbf{Achilles}: I don't know what you mean by "figure" and "ground" in this case. 

\noindent
\textbf{Tortoise}: (Surely you've seen the woodcut 'Mosaic II'?)

\noindent
\textbf{Achilles}: Certainly I know Mosaic II! I know ALL of Escher's works. After all, he's my favorite artist. In any case, I've got a print of Mosaic II hanging on my wall, in plain view from here. 

\noindent
\textbf{Achilles}: Yes, I see all the black animals. 

\noindent
\textbf{Tortoise}: (And do you see the spaces between the black animals?) 

\noindent
\textbf{Achilles}: Yes, I also see how their "negative space" -- what's left out-- defines the white animals. 

\noindent
\textbf{Tortoise}: (Exactly. The ground of one is the figure of the other.) 

\noindent
\textbf{Achilles}: So THAT'S what you mean by "figure" and "ground". But what does that have to do with the "ADAC" puzzle? 

\noindent
\textbf{Tortoise}: (Look at the word 'HEADACHE' again, Achilles. Look at the 'ground' between the 'HE's.) 

\noindent
\textbf{Achilles}: Oh, this is too tricky for me. I think I'M starting to get a headache. 

\noindent
\textbf{Tortoise}: (Perhaps I should come over and we can look at your Escher print together.) 

\noindent
\textbf{Achilles}: You want to come over now? But I thought— 

\noindent
\textbf{Tortoise}: (I think the walk will do my neck some good, and your Bach recording is calling to me.) 

\noindent
\textbf{Achilles}: Very well. Perhaps by then I'll have thought of the right answer to YOUR puzzle, using your figure-ground hint, relating it to MY puzzle. 

\noindent
\textbf{Tortoise}: (I look forward to it. And I look forward to those unaccompanied sonatas.) 

\noindent
\textbf{Achilles}: I'd love to play them for you. 

\noindent
\textbf{Tortoise}: (You know, I have a theory that those sonatas aren't actually unaccompanied.) 

\noindent
\textbf{Achilles}: You've invented a theory about them? 

\noindent
\textbf{Tortoise}: (Yes. I believe Bach intended them to be accompanied by a silent, mental instrument.) 

\noindent
\textbf{Achilles}: Accompanied by what instrument? 

\noindent
\textbf{Tortoise}: (By a harpsichord, of course, but one that plays only in the listener's mind.) 

\noindent
\textbf{Achilles}: Well, if that's the case, it seems a little strange that he would have written out the harpsichord part, then, and had it published as well. 

\noindent
\textbf{Tortoise}: (Ah, but that was only for those with no imagination.) 

\noindent
\textbf{Achilles}: I see -- sort of an optional feature. One could listen to them either way -- with or without accompaniment. But how would one know what the accompaniment is supposed to sound like? 

\noindent
\textbf{Tortoise}: (You hear it in the 'ground'—the spaces between the violin notes.) 

\noindent
\textbf{Achilles}: Ah, yes, I guess that it is best, after all, to leave it to the listener’s imagination. And perhaps, as you said, Bach never even had accompaniment in mind at all. Those sonatas seem to work very indeed as they are. 

\noindent
\textbf{Achilles}: Right. Well, I'll see you shortly. 

\noindent
\textbf{Tortoise}: (Until then, Achilles.) 

\noindent
\textbf{Achilles}: Good-bye, Mr. T.
\end{document}
