\documentclass[11pt]{article}

\usepackage[margin=1in]{geometry}
\usepackage[T1]{fontenc}
\usepackage{lmodern}
\usepackage{microtype}
\usepackage{setspace}
\setstretch{1.08}

\usepackage{enumitem}
\setlist[itemize]{leftmargin=*, itemsep=0.35em, topsep=0.25em}
\setlist[enumerate]{leftmargin=*, itemsep=0.35em, topsep=0.25em}
\usepackage[hidelinks]{hyperref}

\usepackage{titlesec}
\titleformat{\section}{\large\bfseries}{\thesection)}{0.5em}{}
\titlespacing*{\section}{0pt}{1.0em}{0.5em}

\newcommand{\blankline}{\vspace{0.9\baselineskip}}

\begin{document}

\begin{center}
    {\huge \textbf{Notebook X}}\\[0.25em]
    {\large Two-Part Invention :: Chapter 2 Meaning and Form}\\[0.25em]
    21 January 2026
\end{center}

\blankline

%===============================================
\section{Summary}
\begin{itemize}
    \item 
    \item 
    \item 
    \item 
    \item 
\end{itemize}
%===============================================

%===============================================
\section{Dialogue $\leftrightarrow$ Chapter Link}
\noindent
\textit{One paragraph (3--6 sentences) describing the structural trick(s) in the dialogue
(repetition, inversion, canon, self-reference) and the matching concept in the chapter.}
%===============================================

%===============================================
\section{Questions}
\noindent
\textit{Non-trivial and text-anchored; for each, cite page/line or rule.} \\
\textbf{Q1:} \\
\textbf{Q2:} \\
\textbf{Q3:} \\
%===============================================

%===============================================
\section{Worked Examples and Exercises}
%===============================================

%===============================================
\section{Quotes}
\noindent
\textit{Each quote should be no more than a paragraph (or at most two), with page number.
If longer than 3 sentences, you may use \dots\ to excerpt. After each quote, explain why it is uniquely central.}

\subsection*{Passage 1 (p. l.)}
\begin{quote}
\noindent
\emph{Hello}
\end{quote}
\noindent\textbf{Why it matters:}

\subsection*{Passage 2 (p. l.)}
\begin{quote}
\noindent
\emph{Hello}
\end{quote}
\noindent\textbf{Why it matters:}
%===============================================

%===============================================
\section{Key Terms / Mappings}
\noindent
\textit{3--6 new terms or cross-domain connections defined in your own words.}
\begin{itemize}
    \item \textbf{Term/Mapping:}
    \item \textbf{Term/Mapping:}
    \item \textbf{Term/Mapping:}
    \item \textbf{Term/Mapping:}
    \item \textbf{Term/Mapping:}
    \item \textbf{Term/Mapping:}
\end{itemize}
%===============================================

%===============================================
\section{Counter Case or Limitation}
\noindent
\textit{A short example that looks like it fits the topic but doesn't, and why (prefer your own).}
%===============================================

%===============================================
\section{Study Card Sentence}
\noindent
<= 10 word summary
%===============================================

%===============================================
\section{Muddiest Point}
%===============================================

\end{document}
